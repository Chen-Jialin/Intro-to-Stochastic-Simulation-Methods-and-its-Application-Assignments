\documentclass[]{article}
\usepackage{lmodern}
\usepackage{amssymb,amsmath}
\usepackage{ifxetex,ifluatex}
\usepackage{fixltx2e} % provides \textsubscript
\ifnum 0\ifxetex 1\fi\ifluatex 1\fi=0 % if pdftex
  \usepackage[T1]{fontenc}
  \usepackage[utf8]{inputenc}
\else % if luatex or xelatex
  \ifxetex
    \usepackage{mathspec}
  \else
    \usepackage{fontspec}
  \fi
  \defaultfontfeatures{Ligatures=TeX,Scale=MatchLowercase}
\fi
% use upquote if available, for straight quotes in verbatim environments
\IfFileExists{upquote.sty}{\usepackage{upquote}}{}
% use microtype if available
\IfFileExists{microtype.sty}{%
\usepackage{microtype}
\UseMicrotypeSet[protrusion]{basicmath} % disable protrusion for tt fonts
}{}
\usepackage[margin=1in]{geometry}
\usepackage{hyperref}
\hypersetup{unicode=true,
            pdftitle={随机模拟方法与应用导论 作业二},
            pdfauthor={陈稼霖 45875852},
            pdfborder={0 0 0},
            breaklinks=true}
\urlstyle{same}  % don't use monospace font for urls
\usepackage{color}
\usepackage{fancyvrb}
\newcommand{\VerbBar}{|}
\newcommand{\VERB}{\Verb[commandchars=\\\{\}]}
\DefineVerbatimEnvironment{Highlighting}{Verbatim}{commandchars=\\\{\}}
% Add ',fontsize=\small' for more characters per line
\usepackage{framed}
\definecolor{shadecolor}{RGB}{248,248,248}
\newenvironment{Shaded}{\begin{snugshade}}{\end{snugshade}}
\newcommand{\AlertTok}[1]{\textcolor[rgb]{0.94,0.16,0.16}{#1}}
\newcommand{\AnnotationTok}[1]{\textcolor[rgb]{0.56,0.35,0.01}{\textbf{\textit{#1}}}}
\newcommand{\AttributeTok}[1]{\textcolor[rgb]{0.77,0.63,0.00}{#1}}
\newcommand{\BaseNTok}[1]{\textcolor[rgb]{0.00,0.00,0.81}{#1}}
\newcommand{\BuiltInTok}[1]{#1}
\newcommand{\CharTok}[1]{\textcolor[rgb]{0.31,0.60,0.02}{#1}}
\newcommand{\CommentTok}[1]{\textcolor[rgb]{0.56,0.35,0.01}{\textit{#1}}}
\newcommand{\CommentVarTok}[1]{\textcolor[rgb]{0.56,0.35,0.01}{\textbf{\textit{#1}}}}
\newcommand{\ConstantTok}[1]{\textcolor[rgb]{0.00,0.00,0.00}{#1}}
\newcommand{\ControlFlowTok}[1]{\textcolor[rgb]{0.13,0.29,0.53}{\textbf{#1}}}
\newcommand{\DataTypeTok}[1]{\textcolor[rgb]{0.13,0.29,0.53}{#1}}
\newcommand{\DecValTok}[1]{\textcolor[rgb]{0.00,0.00,0.81}{#1}}
\newcommand{\DocumentationTok}[1]{\textcolor[rgb]{0.56,0.35,0.01}{\textbf{\textit{#1}}}}
\newcommand{\ErrorTok}[1]{\textcolor[rgb]{0.64,0.00,0.00}{\textbf{#1}}}
\newcommand{\ExtensionTok}[1]{#1}
\newcommand{\FloatTok}[1]{\textcolor[rgb]{0.00,0.00,0.81}{#1}}
\newcommand{\FunctionTok}[1]{\textcolor[rgb]{0.00,0.00,0.00}{#1}}
\newcommand{\ImportTok}[1]{#1}
\newcommand{\InformationTok}[1]{\textcolor[rgb]{0.56,0.35,0.01}{\textbf{\textit{#1}}}}
\newcommand{\KeywordTok}[1]{\textcolor[rgb]{0.13,0.29,0.53}{\textbf{#1}}}
\newcommand{\NormalTok}[1]{#1}
\newcommand{\OperatorTok}[1]{\textcolor[rgb]{0.81,0.36,0.00}{\textbf{#1}}}
\newcommand{\OtherTok}[1]{\textcolor[rgb]{0.56,0.35,0.01}{#1}}
\newcommand{\PreprocessorTok}[1]{\textcolor[rgb]{0.56,0.35,0.01}{\textit{#1}}}
\newcommand{\RegionMarkerTok}[1]{#1}
\newcommand{\SpecialCharTok}[1]{\textcolor[rgb]{0.00,0.00,0.00}{#1}}
\newcommand{\SpecialStringTok}[1]{\textcolor[rgb]{0.31,0.60,0.02}{#1}}
\newcommand{\StringTok}[1]{\textcolor[rgb]{0.31,0.60,0.02}{#1}}
\newcommand{\VariableTok}[1]{\textcolor[rgb]{0.00,0.00,0.00}{#1}}
\newcommand{\VerbatimStringTok}[1]{\textcolor[rgb]{0.31,0.60,0.02}{#1}}
\newcommand{\WarningTok}[1]{\textcolor[rgb]{0.56,0.35,0.01}{\textbf{\textit{#1}}}}
\usepackage{graphicx,grffile}
\makeatletter
\def\maxwidth{\ifdim\Gin@nat@width>\linewidth\linewidth\else\Gin@nat@width\fi}
\def\maxheight{\ifdim\Gin@nat@height>\textheight\textheight\else\Gin@nat@height\fi}
\makeatother
% Scale images if necessary, so that they will not overflow the page
% margins by default, and it is still possible to overwrite the defaults
% using explicit options in \includegraphics[width, height, ...]{}
\setkeys{Gin}{width=\maxwidth,height=\maxheight,keepaspectratio}
\IfFileExists{parskip.sty}{%
\usepackage{parskip}
}{% else
\setlength{\parindent}{0pt}
\setlength{\parskip}{6pt plus 2pt minus 1pt}
}
\setlength{\emergencystretch}{3em}  % prevent overfull lines
\providecommand{\tightlist}{%
  \setlength{\itemsep}{0pt}\setlength{\parskip}{0pt}}
\setcounter{secnumdepth}{0}
% Redefines (sub)paragraphs to behave more like sections
\ifx\paragraph\undefined\else
\let\oldparagraph\paragraph
\renewcommand{\paragraph}[1]{\oldparagraph{#1}\mbox{}}
\fi
\ifx\subparagraph\undefined\else
\let\oldsubparagraph\subparagraph
\renewcommand{\subparagraph}[1]{\oldsubparagraph{#1}\mbox{}}
\fi

%%% Use protect on footnotes to avoid problems with footnotes in titles
\let\rmarkdownfootnote\footnote%
\def\footnote{\protect\rmarkdownfootnote}

%%% Change title format to be more compact
\usepackage{titling}

% Create subtitle command for use in maketitle
\providecommand{\subtitle}[1]{
  \posttitle{
    \begin{center}\large#1\end{center}
    }
}

\setlength{\droptitle}{-2em}

  \title{随机模拟方法与应用导论 作业二}
    \pretitle{\vspace{\droptitle}\centering\huge}
  \posttitle{\par}
    \author{陈稼霖 45875852}
    \preauthor{\centering\large\emph}
  \postauthor{\par}
      \predate{\centering\large\emph}
  \postdate{\par}
    \date{2019-09-30}

\usepackage[UTF8]{ctex}

\begin{document}
\maketitle

\hypertarget{mammals-data-continued}{%
\section{2.5 (mammals data, continued)}\label{mammals-data-continued}}

Refer to Exercise 2.4. Construct a scatterplot of the ratio
\(r = brain/body\) vs body size for the full mammals data set.

首先计算题目要求的脑体重量比\(r=brain/body\)

\begin{Shaded}
\begin{Highlighting}[]
\KeywordTok{library}\NormalTok{(MASS)}
\NormalTok{r =}\StringTok{ }\NormalTok{mammals}\OperatorTok{$}\NormalTok{brain }\OperatorTok{/}\StringTok{ }\NormalTok{mammals}\OperatorTok{$}\NormalTok{body}
\end{Highlighting}
\end{Shaded}

然后以体重\(body\)为横坐标,以\(r\)为纵坐标,绘制散点图

\begin{Shaded}
\begin{Highlighting}[]
\KeywordTok{plot}\NormalTok{(mammals}\OperatorTok{$}\NormalTok{body,r,}\DataTypeTok{xlab =} \StringTok{'body'}\NormalTok{,}\DataTypeTok{ylab =} \StringTok{'r'}\NormalTok{)}
\end{Highlighting}
\end{Shaded}

\includegraphics{Homework_2_files/figure-latex/unnamed-chunk-2-1.pdf}

\hypertarget{mammals-data-on-original-scale}{%
\section{2.13 (mammals data on original
scale)}\label{mammals-data-on-original-scale}}

Refer to the mammals data in Example 2.7. Construct a scatterplot like
Figure 2.19 on the original scale (Figure 2.19 is on the log-log scale.)
Label the points and distances for cat, cow, and human. In this example,
which plot is easier to interpret?

首先以各哺乳动物的体重\(body\)为横坐标,以各哺乳动物的脑重\(brain\)为纵坐标,绘制散点图;然后计算猫、牛和人之间的``距离''并在图中标出

\begin{Shaded}
\begin{Highlighting}[]
\KeywordTok{plot}\NormalTok{(mammals}\OperatorTok{$}\NormalTok{body,mammals}\OperatorTok{$}\NormalTok{brain,}\DataTypeTok{xlab =} \StringTok{'body'}\NormalTok{,}\DataTypeTok{ylab =} \StringTok{'brain'}\NormalTok{)}
\NormalTok{y =}\StringTok{ }\NormalTok{mammals[}\KeywordTok{c}\NormalTok{(}\StringTok{'Cat'}\NormalTok{,}\StringTok{'Cow'}\NormalTok{,}\StringTok{'Human'}\NormalTok{),]}
\KeywordTok{polygon}\NormalTok{(y)}
\KeywordTok{text}\NormalTok{(y,}\KeywordTok{rownames}\NormalTok{(y))}
\NormalTok{pairs =}\StringTok{ }\KeywordTok{combn}\NormalTok{(}\KeywordTok{rownames}\NormalTok{(y),}\DecValTok{2}\NormalTok{)}
\NormalTok{x =}\StringTok{ }\NormalTok{(y[}\KeywordTok{c}\NormalTok{(pairs[}\DecValTok{1}\NormalTok{,]),] }\OperatorTok{+}\StringTok{ }\NormalTok{y[}\KeywordTok{c}\NormalTok{(pairs[}\DecValTok{2}\NormalTok{,]),]) }\OperatorTok{/}\StringTok{ }\DecValTok{2}
\NormalTok{d =}\StringTok{ }\KeywordTok{as.character}\NormalTok{(}\KeywordTok{dist}\NormalTok{(y))}
\KeywordTok{text}\NormalTok{(x,d)}
\end{Highlighting}
\end{Shaded}

\includegraphics{Homework_2_files/figure-latex/unnamed-chunk-3-1.pdf}

由于数据的数量级差异较大,本题图中散点和标注聚集在左下方,难以分辨,还是例2.7中的对数图(图2.19)更为直观。


\end{document}
