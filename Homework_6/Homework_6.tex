\documentclass[]{article}
\usepackage{lmodern}
\usepackage{amssymb,amsmath}
\usepackage{ifxetex,ifluatex}
\usepackage{fixltx2e} % provides \textsubscript
\ifnum 0\ifxetex 1\fi\ifluatex 1\fi=0 % if pdftex
  \usepackage[T1]{fontenc}
  \usepackage[utf8]{inputenc}
\else % if luatex or xelatex
  \ifxetex
    \usepackage{mathspec}
  \else
    \usepackage{fontspec}
  \fi
  \defaultfontfeatures{Ligatures=TeX,Scale=MatchLowercase}
\fi
% use upquote if available, for straight quotes in verbatim environments
\IfFileExists{upquote.sty}{\usepackage{upquote}}{}
% use microtype if available
\IfFileExists{microtype.sty}{%
\usepackage{microtype}
\UseMicrotypeSet[protrusion]{basicmath} % disable protrusion for tt fonts
}{}
\usepackage[margin=1in]{geometry}
\usepackage{hyperref}
\hypersetup{unicode=true,
            pdftitle={随机模拟方法与应用导论 作业六},
            pdfauthor={陈稼霖 45875852},
            pdfborder={0 0 0},
            breaklinks=true}
\urlstyle{same}  % don't use monospace font for urls
\usepackage{color}
\usepackage{fancyvrb}
\newcommand{\VerbBar}{|}
\newcommand{\VERB}{\Verb[commandchars=\\\{\}]}
\DefineVerbatimEnvironment{Highlighting}{Verbatim}{commandchars=\\\{\}}
% Add ',fontsize=\small' for more characters per line
\usepackage{framed}
\definecolor{shadecolor}{RGB}{248,248,248}
\newenvironment{Shaded}{\begin{snugshade}}{\end{snugshade}}
\newcommand{\AlertTok}[1]{\textcolor[rgb]{0.94,0.16,0.16}{#1}}
\newcommand{\AnnotationTok}[1]{\textcolor[rgb]{0.56,0.35,0.01}{\textbf{\textit{#1}}}}
\newcommand{\AttributeTok}[1]{\textcolor[rgb]{0.77,0.63,0.00}{#1}}
\newcommand{\BaseNTok}[1]{\textcolor[rgb]{0.00,0.00,0.81}{#1}}
\newcommand{\BuiltInTok}[1]{#1}
\newcommand{\CharTok}[1]{\textcolor[rgb]{0.31,0.60,0.02}{#1}}
\newcommand{\CommentTok}[1]{\textcolor[rgb]{0.56,0.35,0.01}{\textit{#1}}}
\newcommand{\CommentVarTok}[1]{\textcolor[rgb]{0.56,0.35,0.01}{\textbf{\textit{#1}}}}
\newcommand{\ConstantTok}[1]{\textcolor[rgb]{0.00,0.00,0.00}{#1}}
\newcommand{\ControlFlowTok}[1]{\textcolor[rgb]{0.13,0.29,0.53}{\textbf{#1}}}
\newcommand{\DataTypeTok}[1]{\textcolor[rgb]{0.13,0.29,0.53}{#1}}
\newcommand{\DecValTok}[1]{\textcolor[rgb]{0.00,0.00,0.81}{#1}}
\newcommand{\DocumentationTok}[1]{\textcolor[rgb]{0.56,0.35,0.01}{\textbf{\textit{#1}}}}
\newcommand{\ErrorTok}[1]{\textcolor[rgb]{0.64,0.00,0.00}{\textbf{#1}}}
\newcommand{\ExtensionTok}[1]{#1}
\newcommand{\FloatTok}[1]{\textcolor[rgb]{0.00,0.00,0.81}{#1}}
\newcommand{\FunctionTok}[1]{\textcolor[rgb]{0.00,0.00,0.00}{#1}}
\newcommand{\ImportTok}[1]{#1}
\newcommand{\InformationTok}[1]{\textcolor[rgb]{0.56,0.35,0.01}{\textbf{\textit{#1}}}}
\newcommand{\KeywordTok}[1]{\textcolor[rgb]{0.13,0.29,0.53}{\textbf{#1}}}
\newcommand{\NormalTok}[1]{#1}
\newcommand{\OperatorTok}[1]{\textcolor[rgb]{0.81,0.36,0.00}{\textbf{#1}}}
\newcommand{\OtherTok}[1]{\textcolor[rgb]{0.56,0.35,0.01}{#1}}
\newcommand{\PreprocessorTok}[1]{\textcolor[rgb]{0.56,0.35,0.01}{\textit{#1}}}
\newcommand{\RegionMarkerTok}[1]{#1}
\newcommand{\SpecialCharTok}[1]{\textcolor[rgb]{0.00,0.00,0.00}{#1}}
\newcommand{\SpecialStringTok}[1]{\textcolor[rgb]{0.31,0.60,0.02}{#1}}
\newcommand{\StringTok}[1]{\textcolor[rgb]{0.31,0.60,0.02}{#1}}
\newcommand{\VariableTok}[1]{\textcolor[rgb]{0.00,0.00,0.00}{#1}}
\newcommand{\VerbatimStringTok}[1]{\textcolor[rgb]{0.31,0.60,0.02}{#1}}
\newcommand{\WarningTok}[1]{\textcolor[rgb]{0.56,0.35,0.01}{\textbf{\textit{#1}}}}
\usepackage{graphicx,grffile}
\makeatletter
\def\maxwidth{\ifdim\Gin@nat@width>\linewidth\linewidth\else\Gin@nat@width\fi}
\def\maxheight{\ifdim\Gin@nat@height>\textheight\textheight\else\Gin@nat@height\fi}
\makeatother
% Scale images if necessary, so that they will not overflow the page
% margins by default, and it is still possible to overwrite the defaults
% using explicit options in \includegraphics[width, height, ...]{}
\setkeys{Gin}{width=\maxwidth,height=\maxheight,keepaspectratio}
\IfFileExists{parskip.sty}{%
\usepackage{parskip}
}{% else
\setlength{\parindent}{0pt}
\setlength{\parskip}{6pt plus 2pt minus 1pt}
}
\setlength{\emergencystretch}{3em}  % prevent overfull lines
\providecommand{\tightlist}{%
  \setlength{\itemsep}{0pt}\setlength{\parskip}{0pt}}
\setcounter{secnumdepth}{0}
% Redefines (sub)paragraphs to behave more like sections
\ifx\paragraph\undefined\else
\let\oldparagraph\paragraph
\renewcommand{\paragraph}[1]{\oldparagraph{#1}\mbox{}}
\fi
\ifx\subparagraph\undefined\else
\let\oldsubparagraph\subparagraph
\renewcommand{\subparagraph}[1]{\oldsubparagraph{#1}\mbox{}}
\fi

%%% Use protect on footnotes to avoid problems with footnotes in titles
\let\rmarkdownfootnote\footnote%
\def\footnote{\protect\rmarkdownfootnote}

%%% Change title format to be more compact
\usepackage{titling}

% Create subtitle command for use in maketitle
\providecommand{\subtitle}[1]{
  \posttitle{
    \begin{center}\large#1\end{center}
    }
}

\setlength{\droptitle}{-2em}

  \title{随机模拟方法与应用导论 作业六}
    \pretitle{\vspace{\droptitle}\centering\huge}
  \posttitle{\par}
    \author{陈稼霖 45875852}
    \preauthor{\centering\large\emph}
  \postauthor{\par}
      \predate{\centering\large\emph}
  \postdate{\par}
    \date{2019-10-04}

\usepackage[UTF8]{ctex}

\begin{document}
\maketitle

\hypertarget{comparing-snowfall-of-buffalo-and-cleveland}{%
\section{6.5 (Comparing snowfall of Buffalo and
Cleveland)}\label{comparing-snowfall-of-buffalo-and-cleveland}}

The datafile \texttt{“buffalo.cleveland.snowfall.txt”} contains the
total snowfall in inches for the cities Buffalo and Cleveland for the
seasons \(1968-69\) through \(2008-09\).

\begin{enumerate}
\def\labelenumi{\alph{enumi}.}
\item
  Compute the differences between the Buffalo snowfall and the Cleveland
  snowfall for all seasons.
\item
  Using the \texttt{t.test} function with the difference data, test the
  hypothesis that Buffalo and Cleveland get, on average, the same total
  snowfall in a season.
\item
  Use the \texttt{t.test} function to construct a \(95\%\) confidence
  interval of the mean difference in seasonal snowfall.
\end{enumerate}

\begin{enumerate}
\def\labelenumi{\alph{enumi}.}
\tightlist
\item
  首先读取文件\texttt{buffalo.cleveland.snowfall.txt}
\end{enumerate}

\begin{Shaded}
\begin{Highlighting}[]
\NormalTok{snowfall =}\StringTok{ }\KeywordTok{read.table}\NormalTok{(}\StringTok{'buffalo.cleveland.snowfall.txt'}\NormalTok{,}\DataTypeTok{head =} \OtherTok{TRUE}\NormalTok{)}
\end{Highlighting}
\end{Shaded}

然后计算各个季度Buffalo和Cleveland的降雪量差值并展示

\begin{Shaded}
\begin{Highlighting}[]
\NormalTok{snowfall}\OperatorTok{$}\NormalTok{diff =}\StringTok{ }\NormalTok{snowfall}\OperatorTok{$}\NormalTok{Buffalo }\OperatorTok{-}\StringTok{ }\NormalTok{snowfall}\OperatorTok{$}\NormalTok{Cleveland}
\NormalTok{snowfall[,}\KeywordTok{c}\NormalTok{(}\DecValTok{1}\NormalTok{,}\DecValTok{4}\NormalTok{)]}
\end{Highlighting}
\end{Shaded}

\begin{verbatim}
##       SEASON  diff
## 1  2008-2009  20.5
## 2  2007-2008  26.6
## 3  2006-2007  12.4
## 4  2005-2006  27.6
## 5  2004-2005  -8.8
## 6  2003-2004   9.7
## 7  2002-2003  15.6
## 8  2001-2002  86.4
## 9  2000-2001  80.6
## 10 1999-2000   3.5
## 11 1998-1999  38.1
## 12 1997-1998  41.6
## 13 1996-1997  41.7
## 14 1995-1996  40.3
## 15 1994-1995  31.0
## 16 1993-1994  40.2
## 17 1992-1993   4.7
## 18 1991-1992  27.1
## 19 1990-1991  10.4
## 20 1989-1990  31.1
## 21 1988-1989  12.6
## 22 1987-1988 -14.9
## 23 1986-1987  11.7
## 24 1985-1986  56.4
## 25 1984-1985  43.5
## 26 1983-1984  53.1
## 27 1982-1983  14.4
## 28 1981-1982  11.9
## 29 1980-1981   0.4
## 30 1979-1980  29.7
## 31 1978-1979  59.0
## 32 1977-1978  64.2
## 33 1976-1977 136.0
## 34 1975-1976  28.1
## 35 1974-1975  28.6
## 36 1973-1974  30.2
## 37 1972-1973  10.3
## 38 1971-1972  64.3
## 39 1970-1971  45.6
## 40 1969-1970  67.1
## 41 1968-1969  41.4
\end{verbatim}

bc.
检验假设---Buffalo和Cleveland季度平均降雪量相等,也就是检验两地的季度降雪量差值的均值为\(0\)。用函数\texttt{t.test}和上面计算得到的差值数据检验该假设,代码和结果如下

\begin{Shaded}
\begin{Highlighting}[]
\KeywordTok{t.test}\NormalTok{(snowfall}\OperatorTok{$}\NormalTok{diff,}\DataTypeTok{mu =} \DecValTok{0}\NormalTok{,}\DataTypeTok{conf.level=}\FloatTok{0.95}\NormalTok{)}
\end{Highlighting}
\end{Shaded}

\begin{verbatim}
## 
##  One Sample t-test
## 
## data:  snowfall$diff
## t = 7.5692, df = 40, p-value = 3.061e-09
## alternative hypothesis: true mean is not equal to 0
## 95 percent confidence interval:
##  24.56221 42.45731
## sample estimates:
## mean of x 
##  33.50976
\end{verbatim}

由结果可知,t检验统计量(t-test
statistic)为\(7.5692\),p值为\(3.061\times10^{-9}\),由于p值很小(\(p\ll0.05\)),故拒绝假设;两地季度降雪量平均差值的\(95\%\)置信区间应为\((24.56221,42.45731)\)。


\end{document}
