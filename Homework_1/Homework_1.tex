\documentclass[]{article}
\usepackage{lmodern}
\usepackage{amssymb,amsmath}
\usepackage{ifxetex,ifluatex}
\usepackage{fixltx2e} % provides \textsubscript
\ifnum 0\ifxetex 1\fi\ifluatex 1\fi=0 % if pdftex
  \usepackage[T1]{fontenc}
  \usepackage[utf8]{inputenc}
\else % if luatex or xelatex
  \ifxetex
    \usepackage{mathspec}
  \else
    \usepackage{fontspec}
  \fi
  \defaultfontfeatures{Ligatures=TeX,Scale=MatchLowercase}
\fi
% use upquote if available, for straight quotes in verbatim environments
\IfFileExists{upquote.sty}{\usepackage{upquote}}{}
% use microtype if available
\IfFileExists{microtype.sty}{%
\usepackage{microtype}
\UseMicrotypeSet[protrusion]{basicmath} % disable protrusion for tt fonts
}{}
\usepackage[margin=1in]{geometry}
\usepackage{hyperref}
\hypersetup{unicode=true,
            pdftitle={随机模拟方法与应用导论 作业一},
            pdfauthor={陈稼霖 45875852},
            pdfborder={0 0 0},
            breaklinks=true}
\urlstyle{same}  % don't use monospace font for urls
\usepackage{color}
\usepackage{fancyvrb}
\newcommand{\VerbBar}{|}
\newcommand{\VERB}{\Verb[commandchars=\\\{\}]}
\DefineVerbatimEnvironment{Highlighting}{Verbatim}{commandchars=\\\{\}}
% Add ',fontsize=\small' for more characters per line
\usepackage{framed}
\definecolor{shadecolor}{RGB}{248,248,248}
\newenvironment{Shaded}{\begin{snugshade}}{\end{snugshade}}
\newcommand{\AlertTok}[1]{\textcolor[rgb]{0.94,0.16,0.16}{#1}}
\newcommand{\AnnotationTok}[1]{\textcolor[rgb]{0.56,0.35,0.01}{\textbf{\textit{#1}}}}
\newcommand{\AttributeTok}[1]{\textcolor[rgb]{0.77,0.63,0.00}{#1}}
\newcommand{\BaseNTok}[1]{\textcolor[rgb]{0.00,0.00,0.81}{#1}}
\newcommand{\BuiltInTok}[1]{#1}
\newcommand{\CharTok}[1]{\textcolor[rgb]{0.31,0.60,0.02}{#1}}
\newcommand{\CommentTok}[1]{\textcolor[rgb]{0.56,0.35,0.01}{\textit{#1}}}
\newcommand{\CommentVarTok}[1]{\textcolor[rgb]{0.56,0.35,0.01}{\textbf{\textit{#1}}}}
\newcommand{\ConstantTok}[1]{\textcolor[rgb]{0.00,0.00,0.00}{#1}}
\newcommand{\ControlFlowTok}[1]{\textcolor[rgb]{0.13,0.29,0.53}{\textbf{#1}}}
\newcommand{\DataTypeTok}[1]{\textcolor[rgb]{0.13,0.29,0.53}{#1}}
\newcommand{\DecValTok}[1]{\textcolor[rgb]{0.00,0.00,0.81}{#1}}
\newcommand{\DocumentationTok}[1]{\textcolor[rgb]{0.56,0.35,0.01}{\textbf{\textit{#1}}}}
\newcommand{\ErrorTok}[1]{\textcolor[rgb]{0.64,0.00,0.00}{\textbf{#1}}}
\newcommand{\ExtensionTok}[1]{#1}
\newcommand{\FloatTok}[1]{\textcolor[rgb]{0.00,0.00,0.81}{#1}}
\newcommand{\FunctionTok}[1]{\textcolor[rgb]{0.00,0.00,0.00}{#1}}
\newcommand{\ImportTok}[1]{#1}
\newcommand{\InformationTok}[1]{\textcolor[rgb]{0.56,0.35,0.01}{\textbf{\textit{#1}}}}
\newcommand{\KeywordTok}[1]{\textcolor[rgb]{0.13,0.29,0.53}{\textbf{#1}}}
\newcommand{\NormalTok}[1]{#1}
\newcommand{\OperatorTok}[1]{\textcolor[rgb]{0.81,0.36,0.00}{\textbf{#1}}}
\newcommand{\OtherTok}[1]{\textcolor[rgb]{0.56,0.35,0.01}{#1}}
\newcommand{\PreprocessorTok}[1]{\textcolor[rgb]{0.56,0.35,0.01}{\textit{#1}}}
\newcommand{\RegionMarkerTok}[1]{#1}
\newcommand{\SpecialCharTok}[1]{\textcolor[rgb]{0.00,0.00,0.00}{#1}}
\newcommand{\SpecialStringTok}[1]{\textcolor[rgb]{0.31,0.60,0.02}{#1}}
\newcommand{\StringTok}[1]{\textcolor[rgb]{0.31,0.60,0.02}{#1}}
\newcommand{\VariableTok}[1]{\textcolor[rgb]{0.00,0.00,0.00}{#1}}
\newcommand{\VerbatimStringTok}[1]{\textcolor[rgb]{0.31,0.60,0.02}{#1}}
\newcommand{\WarningTok}[1]{\textcolor[rgb]{0.56,0.35,0.01}{\textbf{\textit{#1}}}}
\usepackage{graphicx,grffile}
\makeatletter
\def\maxwidth{\ifdim\Gin@nat@width>\linewidth\linewidth\else\Gin@nat@width\fi}
\def\maxheight{\ifdim\Gin@nat@height>\textheight\textheight\else\Gin@nat@height\fi}
\makeatother
% Scale images if necessary, so that they will not overflow the page
% margins by default, and it is still possible to overwrite the defaults
% using explicit options in \includegraphics[width, height, ...]{}
\setkeys{Gin}{width=\maxwidth,height=\maxheight,keepaspectratio}
\IfFileExists{parskip.sty}{%
\usepackage{parskip}
}{% else
\setlength{\parindent}{0pt}
\setlength{\parskip}{6pt plus 2pt minus 1pt}
}
\setlength{\emergencystretch}{3em}  % prevent overfull lines
\providecommand{\tightlist}{%
  \setlength{\itemsep}{0pt}\setlength{\parskip}{0pt}}
\setcounter{secnumdepth}{0}
% Redefines (sub)paragraphs to behave more like sections
\ifx\paragraph\undefined\else
\let\oldparagraph\paragraph
\renewcommand{\paragraph}[1]{\oldparagraph{#1}\mbox{}}
\fi
\ifx\subparagraph\undefined\else
\let\oldsubparagraph\subparagraph
\renewcommand{\subparagraph}[1]{\oldsubparagraph{#1}\mbox{}}
\fi

%%% Use protect on footnotes to avoid problems with footnotes in titles
\let\rmarkdownfootnote\footnote%
\def\footnote{\protect\rmarkdownfootnote}

%%% Change title format to be more compact
\usepackage{titling}

% Create subtitle command for use in maketitle
\providecommand{\subtitle}[1]{
  \posttitle{
    \begin{center}\large#1\end{center}
    }
}

\setlength{\droptitle}{-2em}

  \title{随机模拟方法与应用导论 作业一}
    \pretitle{\vspace{\droptitle}\centering\huge}
  \posttitle{\par}
    \author{陈稼霖 45875852}
    \preauthor{\centering\large\emph}
  \postauthor{\par}
      \predate{\centering\large\emph}
  \postdate{\par}
    \date{2019-09-23}

\usepackage[UTF8]{ctex}

\begin{document}
\maketitle

\hypertarget{binomial-cdf}{%
\section{1.5 (Binomial CDF)}\label{binomial-cdf}}

Let \(X\) be the number of ``ones'' obtained in \(12\) rolls of a fair
die. Then \(X\) has a Binomial(\(n = 12,p = 1/6\)) distribution. Compute
a table of cumulative binomial probabilities (the CDF) for
\(x = 0,1,\dots,12\) by two methods: (1) using \textbf{cumsum} and the
result of Exercise 1.4, and (2) using the \textbf{pbinom} function. What
is \(P(X > 7)\)?

\begin{enumerate}
\def\labelenumi{(\arabic{enumi})}
\tightlist
\item
  先计算二项式分布的PMF \[
  P(X=k)=\left(\begin{array}{c}n\\k\end{array}\right)p^k\left(1-p\right)^{n-k}=\left(\begin{array}{c}12\\k\end{array}\right)\left(\frac{1}{3}\right)^k\left(1-\frac{1}{3}\right)
  \]
\end{enumerate}

\begin{Shaded}
\begin{Highlighting}[]
\KeywordTok{options}\NormalTok{(}\DataTypeTok{digits =} \DecValTok{8}\NormalTok{)}
\NormalTok{n =}\StringTok{ }\DecValTok{12}
\NormalTok{p =}\StringTok{ }\DecValTok{1}\OperatorTok{/}\DecValTok{3}
\NormalTok{k =}\StringTok{ }\KeywordTok{c}\NormalTok{(}\DecValTok{0}\OperatorTok{:}\NormalTok{n)}
\NormalTok{P =}\StringTok{ }\KeywordTok{choose}\NormalTok{(n,k) }\OperatorTok{*}\StringTok{ }\NormalTok{p}\OperatorTok{^}\NormalTok{k }\OperatorTok{*}\StringTok{ }\NormalTok{(}\DecValTok{1} \OperatorTok{-}\StringTok{ }\NormalTok{p)}\OperatorTok{^}\NormalTok{(n }\OperatorTok{-}\StringTok{ }\NormalTok{k)}
\NormalTok{P}
\end{Highlighting}
\end{Shaded}

\begin{verbatim}
##  [1] 7.7073466e-03 4.6244080e-02 1.2717122e-01 2.1195203e-01 2.3844604e-01
##  [6] 1.9075683e-01 1.1127482e-01 4.7689207e-02 1.4902877e-02 3.3117505e-03
## [11] 4.9676258e-04 4.5160234e-05 1.8816764e-06
\end{verbatim}

然后利用函数\textbf{cumsum}计算其CDF

\begin{Shaded}
\begin{Highlighting}[]
\NormalTok{F =}\StringTok{ }\KeywordTok{cumsum}\NormalTok{(P)}
\NormalTok{F_k =}\StringTok{ }\KeywordTok{cbind}\NormalTok{(k,F)}
\KeywordTok{colnames}\NormalTok{(F_k) =}\StringTok{ }\KeywordTok{c}\NormalTok{(}\StringTok{'k'}\NormalTok{,}\StringTok{'F(X=k)'}\NormalTok{)}
\NormalTok{F_k}
\end{Highlighting}
\end{Shaded}

\begin{verbatim}
##        k       F(X=k)
##  [1,]  0 0.0077073466
##  [2,]  1 0.0539514264
##  [3,]  2 0.1811226458
##  [4,]  3 0.3930746781
##  [5,]  4 0.6315207144
##  [6,]  5 0.8222775435
##  [7,]  6 0.9335523605
##  [8,]  7 0.9812415677
##  [9,]  8 0.9961444450
## [10,]  9 0.9994561955
## [11,] 10 0.9999529581
## [12,] 11 0.9999981183
## [13,] 12 1.0000000000
\end{verbatim}

\begin{enumerate}
\def\labelenumi{(\arabic{enumi})}
\setcounter{enumi}{1}
\tightlist
\item
  直接用函数\textbf{pbinom}函数计算二项式分布的CDF
\end{enumerate}

\begin{Shaded}
\begin{Highlighting}[]
\NormalTok{F =}\StringTok{ }\KeywordTok{pbinom}\NormalTok{(k,n,p)}
\NormalTok{F_k =}\StringTok{ }\KeywordTok{cbind}\NormalTok{(k,F)}
\KeywordTok{colnames}\NormalTok{(F_k) =}\StringTok{ }\KeywordTok{c}\NormalTok{(}\StringTok{'k'}\NormalTok{,}\StringTok{'F(X=k)'}\NormalTok{)}
\NormalTok{F_k}
\end{Highlighting}
\end{Shaded}

\begin{verbatim}
##        k       F(X=k)
##  [1,]  0 0.0077073466
##  [2,]  1 0.0539514264
##  [3,]  2 0.1811226458
##  [4,]  3 0.3930746781
##  [5,]  4 0.6315207144
##  [6,]  5 0.8222775435
##  [7,]  6 0.9335523605
##  [8,]  7 0.9812415677
##  [9,]  8 0.9961444450
## [10,]  9 0.9994561955
## [11,] 10 0.9999529581
## [12,] 11 0.9999981183
## [13,] 12 1.0000000000
\end{verbatim}

计算\(P(x>7)\)

\begin{Shaded}
\begin{Highlighting}[]
\NormalTok{F =}\StringTok{ }\DecValTok{1}\OperatorTok{-}\KeywordTok{pbinom}\NormalTok{(}\DecValTok{7}\NormalTok{,n,p)}
\DecValTok{1}\OperatorTok{-}\KeywordTok{pbinom}\NormalTok{(}\DecValTok{7}\NormalTok{,n,p)}
\end{Highlighting}
\end{Shaded}

\begin{verbatim}
## [1] 0.018758432
\end{verbatim}

\hypertarget{lunatics-data}{%
\section{1.13 (lunatics data)}\label{lunatics-data}}

Obtain a five-number summary for the numeric variables in the lunatics
data set (see Example 1.12). From the summary we can get an idea about
the skewness of variables by comparing the median and the mean
population. Which of the distributions are skewed, and in which
direction?

先生成所需数据集

\begin{Shaded}
\begin{Highlighting}[]
\NormalTok{x =}\StringTok{ }\KeywordTok{matrix}\NormalTok{(}\KeywordTok{rnorm}\NormalTok{(}\DecValTok{20}\NormalTok{),}\DecValTok{10}\NormalTok{,}\DecValTok{2}\NormalTok{)}
\NormalTok{x}
\end{Highlighting}
\end{Shaded}

\begin{verbatim}
##              [,1]         [,2]
##  [1,] -0.65822854 -2.272548683
##  [2,]  0.83044396 -0.549542906
##  [3,] -1.16299565  1.535298576
##  [4,] -0.37376462 -1.203134547
##  [5,] -1.24063003  0.056282405
##  [6,] -0.18121616 -0.860101256
##  [7,]  2.76138371  0.417178376
##  [8,]  2.01819901 -0.092126212
##  [9,] -0.73820170 -0.606404806
## [10,] -0.91010968 -1.574098261
\end{verbatim}

生成数据集的summary

\begin{Shaded}
\begin{Highlighting}[]
\KeywordTok{summary}\NormalTok{(x)}
\end{Highlighting}
\end{Shaded}

\begin{verbatim}
##        V1                  V2          
##  Min.   :-1.240630   Min.   :-2.27255  
##  1st Qu.:-0.867133   1st Qu.:-1.11738  
##  Median :-0.515997   Median :-0.57797  
##  Mean   : 0.034488   Mean   :-0.51492  
##  3rd Qu.: 0.577529   3rd Qu.: 0.01918  
##  Max.   : 2.761384   Max.   : 1.53530
\end{verbatim}

从中可见,V1的平均值大于其中位值,因此V1正偏(positive
skewed),V2的平均值大于其中位值,因此V2也正偏。


\end{document}
